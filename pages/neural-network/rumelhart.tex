\section{Многослойный перцептрон Румельхарта}

\indent \indent Многослойный перцептрон — частный случай перцептрона Розенблатта, в котором один алгоритм обратного распространения ошибки обучает все слои. Название по историческим причинам не отражает особенности данного вида перцептрона, то есть не связано с тем, что в нём имеется несколько слоёв (так как несколько слоёв было и у перцептрона Розенблатта). Особенностью является наличие более чем одного обучаемого слоя (как правило — два или три). 

Введем следующие обозначения:
\begin{enumerate}
  \item[1)] $w_{jk}^l$ - вес, который соединяет нейрон с номером $j$ из слоя с номером $l$ c нейроном с номером $k$ из слоя $l-1$;
  \item[2)] $z_j^l$ - результат сумматорной функции нейрона $j$ из слоя $l$;
  \item[3)] $a_j^l$ - результат активационной функции нейрона $j$ из слоя $l$;
  \item[4)] $b_j^l$ - смещение нейрона $j$ из слоя $l$;
  \item[5)] $W^l$ - матрица весов входящих в нейроны в с номером слоя $l$.
\end{enumerate}







