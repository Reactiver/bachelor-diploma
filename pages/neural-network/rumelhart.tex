\section{Многослойный перцептрон Румельхарта}

Многослойный перцептрон — частный случай перцептрона Розенблатта, в котором один алгоритм обратного распространения ошибки обучает все слои. Название по историческим причинам не отражает особенности данного вида перцептрона, то есть не связано с тем, что в нём имеется несколько слоёв (так как несколько слоёв было и у перцептрона Розенблатта). Особенностью является наличие более чем одного обучаемого слоя (как правило — два или три). 

$w_{jk}^l$ - вес, который соединяет нейрон с номером $j$ из слоя с номером $l$ c нейроном с номером $k$ из слоя $l-1$

$z_j^l$ - результат сумматорной функции нейрона $j$ из слоя $l$

$a_j^l$ - результат активационной функции нейрона $j$ из слоя $l$

$b_j^l$ - смещение нейрона $j$ из слоя $l$

$W^l$ - матрица весов входящих в нейроны в с номером слоя $l$