\section{Основные определения}
\indent \indent В данном разделе даны основные определения, связанные с нейронными сетями и эволюционными алгоритмами.

 \textbf{Алгоритм} ~--- Совокупность последовательных шагов, схема действий, приводящих к желаемому результату. 

 \textbf{Генетический алгоритм} ~--- это эвристический алгоритм поиска, используемый для решения задач оптимизации и моделирования путём случайного подбора, комбинирования и вариации искомых параметров с использованием механизмов, аналогичных естественному отбору в природе.

 \textbf{Популяция} ~--- это конечное множество особей.

 \textbf{Нейрон} ~--- узел искусственной нейронной сети, являющийся упрощённой моделью естественного нейрона. Математически, искусственный нейрон обычно представляют как некоторую нелинейную функцию от единственного аргумента — линейной комбинации всех входных сигналов. Данную функцию называют функцией активации. Полученный результат посылается на единственный выход. Такие искусственные нейроны объединяют в сети — соединяют выходы одних нейронов с входами других. 

 \textbf{Искусственная нейронная сеть (ИНС)}  ~--- математическая модель, а также её программное или аппаратное воплощение, построенная по принципу организации и функционирования биологических нейронных сетей — сетей нервных клеток живого организма. После разработки алгоритмов обучения получаемые модели стали использовать в практических целях: в задачах прогнозирования, для распознавания образов, в задачах управления и т.д. 

 \textbf{Эпоха} ~--- одна итерация в процессе обучения, включающая предъявление всех примеров из обучающего множества и, возможно, проверку качества обучения на контрольном множестве. 

 \textbf{Алгоритм обратного распространения ошибки (Back propagation)} ~--- один из методов обучения многослойных нейронных сетей прямого распространения, называемых также многослойными перцептронами.