\section{Обучающие алгоритмы}

\subsection*{Линейная регрессия} 
\indent \indent Алгоритм линейной регрессии используется для оценки реальных значений (стоимость домов, количество звонков, общий объем продаж и.т.д.). На основе непрерывных переменных. Здесь устанавливается связь между независимыми и зависимыми переменными, подбирая лучшую прямую. Эта прямая наилучшего соответствия называется линией регрессии и представлена линейным уравнением $y = ax + b$.
Линейная регрессия бывает двух типов: простая линейная регрессия и множественная линейная регрессия. Простая линейная регрессия характеризуется одной независимой переменной. И, множественная линейная регрессия характеризуется множеством независимых переменных. Найдя наиболее подходящую прямую, можно подобрать полиномиальную или криволинейную регрессию. 

\subsection*{Логистическая регрессия} 
\indent \indent Логистическая регрессия ~--- это статистическая модель, используемая для прогнозирования вероятности возникновения некоторого события путём подгонки данных к логистической кривой. Он используется для оценки дискретных значений (двоичные значения, такие как 0/1, да / нет, истина / ложь) на основе заданного набора независимых переменных. Этот алгоритм предсказывает вероятность возникновения события путем подгонки данных к функции logit . Следовательно, это также известно как регрессия логита . Поскольку он предсказывает вероятность, его выходные значения лежат между 0 и 1.

\subsection*{Наивная байесовская классификация}
\indent \indent Наивная байесовская классификация ~--- Это метод классификации, основанный на  теореме Байеса с предположением независимости между предикторами. Наивный байесовский классификатор предполагает, что наличие определенной функции в классе не связано с наличием любой другой функции. Например, фрукт можно считать яблоком, если оно красное, круглое и около 3 дюймов в диаметре. Даже если эти признаки зависят друг от друга или от наличия других признаков, наивный байесовский классификатор будет рассматривать все эти свойства независимо, чтобы повысить вероятность того, что этот фрукт является яблоком.
Наивная байесовская модель проста в построении и особенно полезна для очень больших наборов данных. Известно, что наряду с простотой наивный байесовский метод превосходит даже самые сложные методы классификации.
