\newpage

\chapter{Нейронные сети}
\section{Применение нейронных сетей}

Нейронные сети используются для решения сложных задач, которые требуют аналитических вычислений подобных тем, что делает человеческий мозг. Самыми распространенными применениями нейронных сетей являются: \\

 \textbf{Классификация} — распределение данных по параметрам. Например, на вход дается набор людей и нужно решить, кому из них давать кредит, а кому нет. Эту работу может сделать нейронная сеть, анализируя такую информацию как: возраст, платежеспособность, кредитная история и т.д. \\

 \textbf{Предсказание} — возможность предсказывать следующий шаг. Например, рост или падение акций, основываясь на ситуации на фондовом рынке. \\

 \textbf{Распознавание} — в настоящее время, самое широкое применение нейронных сетей. Используется в Google, когда пользователи ищут фото или в камерах телефонов, когда оно определяет положение вашего лица и выделяет его и многое другое.

