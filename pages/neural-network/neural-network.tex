\newpage

\chapter{Нейронные сети}
\section{Применение нейронных сетей}

\indent\indent Нейронные сети используются для решения сложных задач, которые требуют аналитических вычислений подобных тем, что делает человеческий мозг. Ниже приведены самые основные задачи, которые решают нейронные сети.

 \textbf{Классификация} — это распределение данных по параметрам. Задано конечное множество объектов, для которых известно, к каким классам они относятся. Это множество называется выборкой. Классовая принадлежность остальных объектов неизвестна. Требуется построить алгоритм, способный классифицировать произвольный объект из исходного множества.

 \textbf{Предсказание} — возможность предсказывать следующий шаг. Например, рост или падение акций, основываясь на ситуации на фондовом рынке. 

 \textbf{Распознавание} — научное направление, связанное с разработкой принципов и построением систем, предназначенных для определения принадлежности данного объекта к одному из заранее выделенных классов объектов.

