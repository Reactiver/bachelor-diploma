\newpage

\chapter{Эволюционные алгоритмы}
\section{Виды эволюционных алгоритмов}

\indent \indent \textbf{Эволюционные алгоритмы} — это направление в компьютерных науках, использующее принципы биологической эволюции для решения задач искусственного интеллекта. Основной принцип биологической эволюции — это сочетание естественного отбора, мутаций и воспроизводства. Хотя эволюционные алгоритмы и пытаются имитировать биологическую эволюцию, они более схожи с искусственным разведением животных, то есть скрещиванием самых лучших представителей, отбором их лучших потомков и повторным скрещиванием уже этих потомков. \\

\textbf{Виды алгоритмов}

\begin{itemize}
  \item Генетические алгоритмы — эвристический алгоритм поиска, используемый для решения задач оптимизации и моделирования путём случайного подбора, комбинирования и вариации искомых параметров;
  \item генетическое программирование — автоматическое создание или изменение программ с помощью генетических алгоритмов;
  \item эволюционное программирование — аналогично генетическому программированию, но структура программы постоянна, изменяются только числовые значения;
  \item эволюционные стратегии — похожи на генетические алгоритмы, но в следующее поколение передаются только положительные мутации;
  \item нейроэволюция — аналогично генетическому программированию, но геномы представляют собой искусственные нейронные сети, в которых происходит эволюция весов при заданной топологии сети, или помимо эволюции весов также производится эволюция топологии.
\end{itemize}
