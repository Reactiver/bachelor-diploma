\section{Применение эволюционных алгоритмов в нейронных сетях}

\indent \indent Разные исследователи высказывали мысль о том, что нейронные сети могут обучаться с помощью генетического алгоритма.
В первых работах на эту тему генетический алгоритм рассматривали в качестве метода обучения небольших однонаправленных нейронных сетей, но в последующем было реализовано применение этого алгоритма для сетей с большей размерностью.

Часто задача заключается в оптимизации весов нейронной сети, имеющей априори заданную топологию. Веса кодируются в виде двоичных последовательностей (хромосом). Каждая особь популяции характеризуется полным множеством весов нейронной сети. Оценка приспособленности особей определяется фитнес функцией, задаваемой в виде суммы квадратов погрешностей.

Существует как минимум два важнейших аргумента в пользу применения генетических алгоритмов для оптимизации весов нейронной сети. Прежде всего, генетические алгоритмы обеспечивают глобальный просмотр пространства весов и позволяют избегать локальные минимумы. Также они могут использоваться в задачах, для которых  очень сложно получить информацию о градиентах либо она оказывается слишком дорогостоящей.

\subsubsection*{Два этапа обучения нейронных сетей}

\indent \indent Эволюционный подход к обучению нейронных сетей состоит из двух основных этапов. Первый этап - это выбор соответствующей схемы представления весов связей. Он заключается в принятии решения - можно ли кодировать эти веса двоичными последовательностями или требуется какая-то другая форма. На втором этапе уже осуществляется сам процесс эволюции, основанный на генетическом алгоритме. В данной работе был выбран метод кодирования матрицы весами нейронной сети.

После выбора схемы хромосомного представления генетический алгоритм применяется к популяции особей с реализацией типового цикла эволюции, состоящего из четырех шагов.

\begin{enumerate}
  \item[1)] Декодирование хромосомы текущего поколения для восстановления множества весов и конструирование соответствующей этому множеству нейронной сети с априорно заданной архитектурой и правилом обучения.
  \item[2)] Расчет общей среднеквадратичной погрешности между фактическими и заданными значениями на всех выходах сети при подаче на ее входы обучающих образов. Погрешность определяет приспособленность сети. Функция приспособленности может быть выбрана в зависимости от задачи.
  \item[3)] Репродукция особей согласно их фитнес функции. Чем выше приспособленность, тем больше шансов имеет особь на скрещивание.
  \item[4)] Применение генетических операторов - таких как скрещивание, мутация и/или инверсия для получения нового поколения.
\end{enumerate}

Достоинством эволюционного подхода считается тот факт, что функцию приспособленности можно легко определить специально для эволюции сети со строго определенными свойствами. Например, если для оценивания приспособленности использовать результаты тестирования вместо результатов обучения, то будет получена сеть с лучшей способностью к обобщению.