\newpage

\chapter*{Заключение}
\addcontentsline{toc}{chapter}{Заключение}

  \indent \indent В настоящее время приобретает популярность такая технология, как нейронные сети. Это происходит благодаря их обучаемости и решению проблем, с которыми не справляются обычные алгоритмы.

  В данной выпускной квалификационной работе были рассмотрены такие виды нейронов, как перцептрон, сигмоидальный нейрон и многослойный перцептрон Румельхарта. Были выведены основные формулы для их обучения. Также было рассмотрено применение эволюционных алгоритмов в нейронных сетях. В ходе исследования был разработан алгоритм скрещивания и эволюционный алгоритм.

  В данной работе была разработана программа, решающая задачу классификации. Программа определяет, попадет ли футболист в мишень только по исходным данным. Были проведены опыты, которые показали наилучшую архитектуру для сети. Проведенные тесты показали хорошую обучаемость нейронной сети с применением эволюционного алгоритма.