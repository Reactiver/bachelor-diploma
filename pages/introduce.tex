\newpage

\chapter*{Введение}
\addcontentsline{toc}{chapter}{Введение}
\section*{Актуальность}
\addcontentsline{toc}{section}{Актуальность}

\indent\indent 
В настоящее время существует ряд задач, для решения которых сложно или невозможно создать алгоритм в традиционном виде. В таком случае используют нейронные сети. Главное преимущество нейронных сетей ~--- это обучаемость. В процессе
обучения искусственная нейронная сеть способна выявлять сложные зависимости
между входными и выходными данными, а также выполнять обобщение. Это
значит, что в случае успешного обучения сеть сможет вернуть правильный
результат на основании данных, которые отсутствовали в обучающей выборке, а
также неполных и/или «зашумленных», частично искажённых данных.

Тема искусственного интеллекта на сегодняшний день является 
очень актуальной. В наши дни создаются основополагающие 
концепции и алгоритмы, связанные с машинным обучением.  
Сегодня существует определенный класс актуальных задач, 
решение которых без применения искусственных нейронных 
сетей (ИНС) невозможно или трудноосуществимо. Зачастую 
сюда относятся такие задачи, как классификация, 
прогнозирование, автоматизация процесса распознавания и автоматизация процесса принятия решений.

\subsection*{Цели и задачи работы}
\addcontentsline{toc}{section}{Цели и задачи работы}
\subsubsection*{Цель работы}

\indent\indent  Разработать алгоритм, реализующий работу эволюционного алгоритма в нейронных сетях.

\subsubsection*{Задачи работы}
\begin{itemize}
	\item изучить обучающие алгоритмы;
	\item изучить виды нейронных сетей и методы их обучения;
	\item изучить применение эволюционных алгоритмов в нейронных сетях;
	\item разработать алгоритм, реализующий работу эволюционного 
  алгоритма в нейронных сетях;
\end{itemize}


\subsection*{Основные результаты}
\addcontentsline{toc}{section}{Основные результаты}

\indent \indent Во время создания выпускной квалификационной работы были достигнуты следующие результаты:
\begin{itemize}
	\item изучены обучающие алгоритмы;
	\item изучены виды нейронных сетей и методы их обучения;
	\item изучено применение эволюционных алгоритмов в нейронных сетях;
	\item разработана нейронная сеть, обучающаяся с помощью эволюционного алгоритма, которая способна решать задачу классификации.
\end{itemize}

\subsection*{Структура работы}
\addcontentsline{toc}{section}{Структура работы}

\indent \indent В первой главе ВКР рассмотрены нейронные сети, их виды, применение и способы обучения. Даны основные понятия и определения. Приведены различные виды обучающих алгоритмов и их применение. Во второй главе рассмотрены эволюционные алгоритмы и применение их в нейронных сетях.В третьей главе рассказывается про разработанное программное обеспечение, про выбор средств разработки, а также про разработанный эволюционный алгоритм и алгоритм скрещивания.
