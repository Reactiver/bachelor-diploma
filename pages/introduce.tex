\newpage

\chapter*{Введение}
\addcontentsline{toc}{chapter}{Введение}
\subsection*{Актуальность}
В последнее время современная техника становится настолько
близка к человеку, что способна понимать его с полуслова.
Трудно представить современный телефон или цифровую камеру
 без функции распознавания лиц. Различные рекламные сервисы
, размещенные в интернете, настолько точно предоставляют
нужную рекламу пользователю, насколько это возможно.  
Поисковые запросы в Google почти всегда дают самые нужные
 ссылки и информацию, основываясь на результатах работы нейронных 
 сетей. Сегодня можно разговаривать с компьютером, и он будет 
 разумно поддерживать диалог. Новейшие технологии в медицине 
 позволяют предсказывать диагноз у пациентов. Все это стало 
 возможным благодаря развитию машинного обучения, в том числе
  благодаря нейронным сетям и эволюционным алгоритмам. 
  Концепция существует уже несколько десятилетий, но в 
  последние годы приобрела огромную популярность благодаря
 передовым технологиям и аппаратному обеспечению. \\ \\
Тема искусственного интеллекта на сегодняшний день является 
очень актуальной. В наши дни создаются основополагающие 
концепции и алгоритмы, связанные с машинным обучением. 
Не случайно самому распространенному алгоритму обучения 
нейронных сетей – алгоритму обратного распространения 
ошибки, нет даже и 10 лет. Также за последние 8 лет интерес 
к машинному обучению возрастает экспоненциально, 
что подтверждает официальная статистика Google. 
Сегодня существует определенный класс актуальных задач, 
решение которых без применения искусственных нейронных 
сетей (ИНС) невозможно или трудноосуществимо. Зачастую 
сюда относятся такие задачи, как классификация, 
прогнозирование и управление сложными системами. 
В последнее время быстро набирает популярность 
концепция глубокого обучения. По сути дела, в этой 
концепции нет ничего революционного, работы по изучению 
искусственных нейронных сетей ведутся с середины прошлого
 века, однако в последнее время уровень производительности 
 персональных вычислительных средств и развитие параллельных 
 вычислительных архитектур позволяет широкому кругу 
 исследователей применять данные структуры более эффективно 
 в области машинного обучения, это, в свою очередь, 
 подстегивает очередной скачок интереса к нейронным сетям. 
 Есть определенные основания полагать, что этот скачок окажется 
 значительным и будет определять концепции развития технологий 
 машинного обучения в дальнейшем.

\subsection*{Цель работы}
Разработать алгоритм, реализующий работу эволюционного алгоритма в нейронных сетях.

\subsection*{Задачи работы}
\begin{itemize}
	\item изучить обучающие алгоритмы;
	\item изучить виды нейронных сетей и методы их обучения;
	\item изучить применение эволюционных алгоритмов в нейронных сетях;
	\item разработать алгоритм, реализующий работу эволюционного 
  алгоритма в нейронных сетях;
\end{itemize}


\subsection*{Основные результаты}
Во время создания выпускной квалификационной работы были исследованы такие понятия как:
нейронные сети;
эволюционные алгоритмы;
алгоритмы обучения нейронных сетей;
обучение нейросетей с помощью эволюционных алгоритмов.
Разработана нейронная сеть, обучающаяся с помощью эволюционного алгоритма, которая способна решать задачу классификации и давать правильный ответ.

\subsection*{Структура работы}
В первой главе ВКР будут рассмотрены нейронные сети, их виды, применение и способы обучения. Вторая глава будет про эволюционные алгоритмы и применение их в нейронных сетях.В третьей главе пойдет речь о результатах работы, о разработанном программном обеспечении.
