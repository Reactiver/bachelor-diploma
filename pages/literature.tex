\newpage

\addcontentsline{toc}{chapter}{Литература}
\begin{thebibliography}{99}

  \bibitem{neuro} Барский А. Нейронные сети: распознавание, управление, принятие решений. — М.: Финансы и статистика, 2004. — 176 с: ил. — (Прикладные информационные технологии). ISBN 5-279-02757-Х.

  \bibitem{Python} Вандер Дж. Python для сложных задач: наука о данных и машинное обучение. — СПб.: Питер, 2018. — 576 с.: ил. — (Серия «Бестселлеры O’Reilly»). ISBN 978-5-496-03068-7. 

  \bibitem{genetic} Гладков Л., Курейчик В. Генетические алгоритмы — М.: Издательская фирма «Физико-математическая литература», 2010. — 367 с: ил. ISBN: 978-5-9221-0510-1.

  \bibitem{reg}	Дрейпер Н., Смит Г. Прикладной регрессионный анализ / Пер. с англ. – М.: Издательский дом «Вильямс», 2007. – 912 c.: ил. ISBN: 978-5-8459-0963-3.

  \bibitem{neuroAndGen} Рутковская Д., Пилиньский М., Рутковский Л. Нейронные сети, генетические алгоритмы и нечеткие системы: Пер. с польск.  И. Д. Рудинского. – 2-е изд., стереотип. –  М.: Горячая линия – Телеком, 2013. – 384 c.: ил. ISBN 978-5-9912-0320-3.

  \bibitem{ns} Хайкин С. Нейронные сети. Полный курс. 2-e изд. Пер. с англ. – М.: Издательский дом "Вильямс", 2006. – 1104 с.: ил. ISBN: 978-5-8459-0890-2.

  \bibitem{sholle} Шолле Ф. Глубокое обучение на Python 
  СПб.: Питер, 2018. — 400 с.: ил. — (Серия «Библиотека программиста»). ISBN 978-5-4461-0770-4.

\end{thebibliography}